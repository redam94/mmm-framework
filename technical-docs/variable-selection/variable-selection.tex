\documentclass[11pt,letterpaper]{article}
\usepackage[margin=1.25in]{geometry}
\usepackage{palatino}
\usepackage{microtype}
\usepackage{parskip}
\usepackage{xcolor}
\usepackage{titlesec}
\usepackage{fancyhdr}
\usepackage{enumitem}
\usepackage{booktabs}
\usepackage{amsmath}
\usepackage{amssymb}
\usepackage{graphicx}
\usepackage{hyperref}
\usepackage{listings}
\usepackage[round]{natbib}
\bibliographystyle{plainnat}

% Define colors
\definecolor{accentblue}{RGB}{0,70,127}
\definecolor{darkgray}{RGB}{64,64,64}
\definecolor{alertred}{RGB}{180,0,0}
\definecolor{warningyellow}{RGB}{200,150,0}
\definecolor{codegreen}{RGB}{0,120,0}

% Section formatting
\titleformat{\section}
  {\Large\bfseries\color{accentblue}}
  {\thesection}
  {1em}
  {}

\titleformat{\subsection}
  {\large\bfseries\color{darkgray}}
  {\thesubsection}
  {1em}
  {}

\titleformat{\subsubsection}
  {\normalsize\bfseries\color{darkgray}}
  {\thesubsubsection}
  {1em}
  {}

% Header/footer
\pagestyle{fancy}
\fancyhf{}
\renewcommand{\headrulewidth}{0.4pt}
\fancyhead[L]{\small\color{darkgray} Variable Selection for Precision Controls}
\fancyhead[R]{\small\color{darkgray} Technical Documentation}
\fancyfoot[C]{\thepage}

% Hyperref setup
\hypersetup{
    colorlinks=true,
    linkcolor=accentblue,
    urlcolor=accentblue
}

% Code listings
\lstset{
    basicstyle=\ttfamily\small,
    keywordstyle=\color{accentblue}\bfseries,
    commentstyle=\color{codegreen},
    stringstyle=\color{alertred},
    breaklines=true,
    frame=single,
    backgroundcolor=\color{gray!5},
}

% Warning box
\newcommand{\warningbox}[1]{%
\begin{center}
\fcolorbox{alertred}{alertred!10}{%
\begin{minipage}{0.9\textwidth}
\textcolor{alertred}{\textbf{CAUTION:}} #1
\end{minipage}
}
\end{center}
}

% Info box
\newcommand{\infobox}[1]{%
\begin{center}
\fcolorbox{accentblue}{accentblue!10}{%
\begin{minipage}{0.9\textwidth}
\textcolor{accentblue}{\textbf{Note:}} #1
\end{minipage}
}
\end{center}
}

\begin{document}

\begin{center}
{\LARGE\bfseries\color{accentblue} Bayesian Variable Selection for Precision Controls}\\[0.5em]
{\large Regularized Horseshoe and Spike-and-Slab Priors in Marketing Mix Models}\\[1.5em]
{\small TECHNICAL DOCUMENTATION}\\[0.5em]
{\small \today}
\end{center}

\vspace{1em}

\section*{Executive Summary}

This document describes the implementation of Bayesian variable selection methods for \textbf{precision control variables} in Marketing Mix Models. These methods---regularized horseshoe, Finnish horseshoe, spike-and-slab, and Bayesian LASSO---provide principled approaches to handling uncertainty about which control variables should be included in a model.

\warningbox{Variable selection priors should \textbf{only} be applied to precision control variables---variables that improve estimation precision but are not confounders, mediators, or otherwise on the causal path between media and outcomes. Applying these methods incorrectly can \textbf{severely bias causal effect estimates}.}

\tableofcontents
\newpage

\section{Motivation and Use Cases}

\subsection{The Problem: Uncertainty About Control Inclusion}

In MMM practice, analysts often face uncertainty about which control variables to include:

\begin{itemize}[leftmargin=1.5em, itemsep=0.3em]
    \item Weather variables may or may not affect foot traffic for a particular retailer
    \item Economic indicators like gas prices may have varying relevance by category
    \item Competitor activities may or may not be adequately captured in the data
    \item Calendar effects (holidays, events) may have heterogeneous impacts
\end{itemize}

Traditional approaches handle this through:
\begin{enumerate}[leftmargin=1.5em, itemsep=0.3em]
    \item \textbf{Include all}: Risk overfitting, especially with many controls
    \item \textbf{Stepwise selection}: Invalidates inference through specification shopping
    \item \textbf{Domain expertise}: Subjective, non-reproducible, may miss relevant variables
    \item \textbf{Regularization (Ridge/LASSO)}: Point estimates only, no uncertainty quantification
\end{enumerate}

\subsection{The Solution: Bayesian Variable Selection}

Bayesian variable selection priors provide a principled alternative:

\begin{itemize}[leftmargin=1.5em, itemsep=0.3em]
    \item \textbf{Adaptive shrinkage}: Small effects shrunk toward zero; large effects preserved
    \item \textbf{Uncertainty quantification}: Full posterior distributions, not point estimates
    \item \textbf{Pre-specification}: Prior beliefs encoded before seeing results
    \item \textbf{Soft selection}: No hard include/exclude decisions; continuous shrinkage
\end{itemize}

\subsection{Appropriate Use Cases}

Variable selection is appropriate for \textbf{precision control variables}:

\begin{table}[h]
\centering
\begin{tabular}{lcc}
\toprule
\textbf{Variable Type} & \textbf{Variable Selection?} & \textbf{Rationale} \\
\midrule
Weather (temperature, precipitation) & \checkmark Yes & Precision variable \\
Economic indicators (gas prices, CPI) & \checkmark Yes & Precision variable \\
Calendar effects (minor holidays) & \checkmark Yes & Precision variable \\
Distribution/ACV & \textbf{$\times$ No} & Potential confounder \\
Price/promotion & \textbf{$\times$ No} & Causal variable \\
Competitive media & \textbf{$\times$ No} & Potential confounder \\
Trend/seasonality & \textbf{$\times$ No} & Core model component \\
\bottomrule
\end{tabular}
\caption{Variable selection applicability by control type}
\end{table}

\newpage
\section{Causal Considerations}

\warningbox{This section is \textbf{critical}. Misapplication of variable selection can introduce severe bias into media effect estimates. Read carefully before implementing.}

\subsection{The Causal Structure of MMM}

Consider the simplified causal graph for MMM:

\begin{center}
\begin{tabular}{c}
\includegraphics[width=0.6\textwidth]{dag_placeholder.png}
\end{tabular}
\end{center}

\noindent \textit{(Note: A proper DAG would show: Media $\rightarrow$ Sales, Confounders $\rightarrow$ Media and Confounders $\rightarrow$ Sales, Precision Variables $\rightarrow$ Sales only)}

Variables fall into distinct causal categories:

\begin{enumerate}[leftmargin=1.5em, itemsep=0.5em]
    \item \textbf{Treatment (Media)}: The causal effect we want to estimate
    \item \textbf{Outcome (Sales)}: What we're modeling
    \item \textbf{Confounders}: Affect \textit{both} media spending and sales
    \begin{itemize}[leftmargin=1em]
        \item Example: Seasonality drives both media planning and consumer behavior
        \item \textit{Must} be controlled for to identify causal effects
    \end{itemize}
    \item \textbf{Precision variables}: Affect sales \textit{only}, not media spending
    \begin{itemize}[leftmargin=1em]
        \item Example: Daily weather affects store visits but not media budgets
        \item Inclusion improves precision but doesn't affect consistency
    \end{itemize}
    \item \textbf{Mediators}: On the causal path from media to sales
    \begin{itemize}[leftmargin=1em]
        \item Example: Brand awareness (media $\rightarrow$ awareness $\rightarrow$ sales)
        \item Controlling for mediators blocks the causal effect
    \end{itemize}
    \item \textbf{Colliders}: Caused by both media and sales
    \begin{itemize}[leftmargin=1em]
        \item Example: Survey response (respondents who both saw ads and bought)
        \item Controlling for colliders introduces spurious associations
    \end{itemize}
\end{enumerate}

\subsection{Why Variable Selection Must Be Restricted}

\subsubsection{Confounders Must Be Included}

If a confounder $C$ affects both media $M$ and sales $Y$:
\begin{equation}
M \leftarrow C \rightarrow Y
\end{equation}

Failing to control for $C$ creates omitted variable bias:
\begin{equation}
\text{Bias}(\hat{\beta}_M) = \beta_C \cdot \frac{\text{Cov}(M, C)}{\text{Var}(M)}
\end{equation}

If variable selection shrinks $\beta_C$ toward zero (because $C$'s direct effect on $Y$ is small), the bias in $\hat{\beta}_M$ remains. The confounder's \textit{correlation} with media, not its \textit{effect} on sales, determines bias.

\textbf{Example}: Suppose seasonality has a modest direct effect on sales (captured mostly by trend/Fourier terms) but strongly drives media flighting. Variable selection might shrink the seasonality control, leaving confounding bias in media estimates.

\subsubsection{Mediators Must Be Excluded}

If brand awareness $A$ mediates the effect of media $M$ on sales $Y$:
\begin{equation}
M \rightarrow A \rightarrow Y
\end{equation}

Controlling for $A$ blocks the indirect effect, leading to underestimation of media's total effect. Variable selection on mediators could partially block causal paths.

\subsubsection{Colliders Must Be Excluded}

If both media $M$ and sales $Y$ cause variable $Z$:
\begin{equation}
M \rightarrow Z \leftarrow Y
\end{equation}

Controlling for $Z$ induces spurious correlation between $M$ and $Y$. Variable selection that partially controls for $Z$ creates partial collider bias.

\subsection{Safe Application: Precision Variables Only}

Variable selection is safe when restricted to \textbf{precision variables}---variables that:
\begin{itemize}[leftmargin=1.5em, itemsep=0.3em]
    \item Affect the outcome (sales)
    \item Do \textbf{not} affect treatment (media spending)
    \item Are \textbf{not} affected by treatment or outcome
\end{itemize}

For such variables, inclusion improves efficiency (tighter posteriors) but non-inclusion doesn't bias causal estimates.

\warningbox{When in doubt about a variable's causal role, \textbf{do not apply variable selection}. Include it with a standard (non-shrinkage) prior, or exclude it entirely based on pre-specified criteria.}

\newpage
\section{Method Descriptions}

\subsection{Regularized Horseshoe Prior}

The regularized horseshoe \citep{piironen2017sparsity} is the recommended default for most applications.

\subsubsection{Mathematical Formulation}

For coefficient $\beta_j$ with $j = 1, \ldots, D$ control variables:

\begin{align}
\beta_j &= z_j \cdot \tau \cdot \tilde{\lambda}_j \\
\tilde{\lambda}_j &= \frac{c \cdot \lambda_j}{\sqrt{c^2 + \tau^2 \lambda_j^2}} \\
\lambda_j &\sim \text{HalfStudentT}(\nu_\lambda) \quad \text{(local shrinkage)} \\
\tau &\sim \text{HalfStudentT}(\nu_\tau, \tau_0) \quad \text{(global shrinkage)} \\
c^2 &\sim \text{InverseGamma}\left(\frac{\nu_c}{2}, \frac{\nu_c s^2}{2}\right) \quad \text{(slab)} \\
z_j &\sim \text{Normal}(0, 1)
\end{align}

where the global shrinkage scale $\tau_0$ is set based on expected sparsity:
\begin{equation}
\tau_0 = \frac{D_0}{D - D_0} \cdot \frac{\sigma}{\sqrt{N}}
\end{equation}

with $D_0$ = expected number of nonzero coefficients, $D$ = total coefficients, $N$ = observations.

\subsubsection{Interpretation}

\begin{itemize}[leftmargin=1.5em, itemsep=0.3em]
    \item $\tau$: Global shrinkage---controls overall sparsity
    \item $\lambda_j$: Local shrinkage---allows individual coefficients to escape shrinkage
    \item $c$: Slab scale---prevents non-zero coefficients from being too large
    \item $\tilde{\lambda}_j$: Regularized local shrinkage---bounded version of $\tau \cdot \lambda_j$
\end{itemize}

\subsubsection{Hyperparameter Recommendations}

\begin{table}[h]
\centering
\begin{tabular}{lll}
\toprule
\textbf{Parameter} & \textbf{Default} & \textbf{Guidance} \\
\midrule
$D_0$ (expected\_nonzero) & 3 & Your prior belief about \# relevant controls \\
$s$ (slab\_scale) & 2.0 & Max expected coefficient magnitude (in std units) \\
$\nu_c$ (slab\_df) & 4.0 & Lower = heavier tails, larger effects allowed \\
$\nu_\lambda$ (local\_df) & 5.0 & Lower = more aggressive shrinkage \\
$\nu_\tau$ (global\_df) & 1.0 & Standard half-Cauchy \\
\bottomrule
\end{tabular}
\caption{Regularized horseshoe hyperparameters}
\end{table}

\subsection{Finnish Horseshoe Prior}

The Finnish horseshoe is mathematically identical to the regularized horseshoe. The name emphasizes the slab regularization component and comes from the Finnish authors \citep{piironen2017sparsity}.

Use the same implementation and hyperparameters as the regularized horseshoe.

\subsection{Spike-and-Slab Prior}

The spike-and-slab \citep{george1993variable} uses a discrete mixture of two distributions.

\subsubsection{Mathematical Formulation}

\begin{align}
\beta_j &= \gamma_j \cdot \beta_j^{\text{slab}} + (1 - \gamma_j) \cdot \beta_j^{\text{spike}} \\
\gamma_j &\sim \text{Bernoulli}(\pi) \\
\beta_j^{\text{slab}} &\sim \text{Normal}(0, \sigma_{\text{slab}}^2) \\
\beta_j^{\text{spike}} &\sim \text{Normal}(0, \sigma_{\text{spike}}^2)
\end{align}

where $\sigma_{\text{spike}} \ll \sigma_{\text{slab}}$ (e.g., 0.01 vs 1.0).

\subsubsection{Continuous Relaxation}

For gradient-based samplers (NUTS), we use a continuous relaxation:
\begin{align}
\gamma_j &= \sigma\left(\frac{\text{logit}_j}{T}\right) \\
\text{logit}_j &\sim \text{Normal}(\text{logit}(\pi), 1)
\end{align}

where $T$ is a temperature parameter controlling sharpness of selection.

\subsubsection{Hyperparameter Recommendations}

\begin{table}[h]
\centering
\begin{tabular}{lll}
\toprule
\textbf{Parameter} & \textbf{Default} & \textbf{Guidance} \\
\midrule
$\pi$ (prior\_inclusion\_prob) & 0.5 & Prior probability of inclusion \\
$\sigma_{\text{spike}}$ (spike\_scale) & 0.01 & Should be small (near-zero) \\
$\sigma_{\text{slab}}$ (slab\_scale) & 1.0 & Expected magnitude of true effects \\
$T$ (temperature) & 0.1 & Lower = sharper selection \\
\bottomrule
\end{tabular}
\caption{Spike-and-slab hyperparameters}
\end{table}

\subsection{Bayesian LASSO}

The Bayesian LASSO \citep{park2008bayesian} uses a Laplace prior, represented as a scale mixture:

\begin{align}
\beta_j | \tau_j &\sim \text{Normal}(0, \tau_j) \\
\tau_j &\sim \text{Exponential}(\lambda^2 / 2)
\end{align}

This is equivalent to a Laplace prior: $\beta_j \sim \text{Laplace}(0, 1/\lambda)$.

\subsubsection{Comparison with Horseshoe}

\begin{itemize}[leftmargin=1.5em, itemsep=0.3em]
    \item LASSO shrinks all coefficients uniformly toward zero
    \item Horseshoe allows large coefficients to escape shrinkage
    \item Horseshoe generally preferred for variable selection
    \item LASSO useful when expecting many small effects (not sparse)
\end{itemize}

\newpage
\section{Implementation}

\subsection{Configuration}

\begin{lstlisting}[language=Python, caption={Variable selection configuration}]
from variable_selection import (
    VariableSelectionConfig,
    VariableSelectionMethod,
    HorseshoeConfig,
)

# Regularized horseshoe (recommended)
config = VariableSelectionConfig(
    method=VariableSelectionMethod.REGULARIZED_HORSESHOE,
    horseshoe=HorseshoeConfig(
        expected_nonzero=3,  # Prior: ~3 controls are relevant
        slab_scale=2.0,      # Max ~2 std effect size
        slab_df=4.0,         # Moderate tail weight
    ),
    # IMPORTANT: Exclude known confounders from selection
    exclude_variables=["distribution", "price", "competitor_media"],
)
\end{lstlisting}

\subsection{Model Integration}

\begin{lstlisting}[language=Python, caption={Integrating with model building}]
import pymc as pm
from variable_selection import create_control_selection_prior

with pm.Model(coords={"control": control_names}) as model:
    # Observation noise (needed for horseshoe scaling)
    sigma = pm.HalfNormal("sigma", sigma=0.5)
    
    # Variable selection prior for controls
    result = create_control_selection_prior(
        name="beta_controls",
        n_variables=len(control_names),
        n_obs=n_observations,
        sigma=sigma,
        config=config,
        dims="control",
    )
    
    # Use result.beta in the model
    control_contribution = pm.math.dot(X_controls, result.beta)
    
    # ... rest of model specification
\end{lstlisting}

\subsection{Diagnostics}

\begin{lstlisting}[language=Python, caption={Post-fit diagnostics}]
from variable_selection import summarize_variable_selection

# After fitting
summary = summarize_variable_selection(
    trace=idata,
    control_names=control_names,
    config=config,
    name="beta_controls",
)

print(summary)
# Shows: variable, mean, std, HDI, inclusion_prob, selected
\end{lstlisting}

\newpage
\section{Best Practices}

\subsection{Pre-Specification Checklist}

Before applying variable selection:

\begin{enumerate}[leftmargin=1.5em, itemsep=0.5em]
    \item \textbf{Classify all variables} into causal categories (confounder, precision, etc.)
    \item \textbf{Document rationale} for each classification
    \item \textbf{Exclude from selection}: all confounders, mediators, colliders
    \item \textbf{Specify hyperparameters} based on domain knowledge, not data
    \item \textbf{Set expected\_nonzero} conservatively (err toward larger values)
\end{enumerate}

\subsection{Hyperparameter Selection}

\begin{itemize}[leftmargin=1.5em, itemsep=0.3em]
    \item \textbf{expected\_nonzero}: If unsure, set to $D/2$ (weak sparsity prior)
    \item \textbf{slab\_scale}: Consider the scale of your outcome; 2.0 works for standardized data
    \item \textbf{Do not} tune hyperparameters based on fit metrics---this is specification shopping
\end{itemize}

\subsection{Reporting Results}

When reporting models with variable selection:

\begin{enumerate}[leftmargin=1.5em, itemsep=0.3em]
    \item State which variables were subject to selection
    \item Report inclusion probabilities, not just point estimates
    \item Show sensitivity to expected\_nonzero specification
    \item Acknowledge uncertainty about variable importance
\end{enumerate}

\subsection{What NOT to Do}

\begin{itemize}[leftmargin=1.5em, itemsep=0.3em]
    \item[\textcolor{alertred}{$\times$}] Apply selection to confounders or core controls
    \item[\textcolor{alertred}{$\times$}] Tune hyperparameters to improve fit or coefficient signs
    \item[\textcolor{alertred}{$\times$}] Use selection as a substitute for thoughtful model specification
    \item[\textcolor{alertred}{$\times$}] Interpret zero coefficients as ``proof'' of no effect
    \item[\textcolor{alertred}{$\times$}] Apply selection to media variables (unless comparing channel subsets)
\end{itemize}

\newpage
\section{Mathematical Details}

\subsection{Shrinkage Factor Interpretation}

The effective shrinkage for coefficient $\beta_j$ in the horseshoe is:
\begin{equation}
\kappa_j = \frac{1}{1 + \tau^2 \lambda_j^2}
\end{equation}

where $\kappa_j \approx 1$ implies strong shrinkage (toward zero) and $\kappa_j \approx 0$ implies weak shrinkage (coefficient preserved).

The \textbf{effective number of nonzero coefficients} is:
\begin{equation}
m_{\text{eff}} = \sum_{j=1}^{D} (1 - \kappa_j)
\end{equation}

This provides a continuous measure of model complexity.

\subsection{Global Shrinkage Calibration}

The global shrinkage $\tau$ controls overall sparsity. The recommended scale:
\begin{equation}
\tau_0 = \frac{p_0}{1 - p_0} \cdot \frac{\sigma}{\sqrt{N}}
\end{equation}

where $p_0 = D_0/D$ is the prior probability that a coefficient is nonzero.

This calibration ensures:
\begin{itemize}[leftmargin=1.5em, itemsep=0.3em]
    \item Prior expected number of nonzero coefficients $\approx D_0$
    \item Shrinkage scales appropriately with sample size
    \item Coefficients are measured in the right units relative to $\sigma$
\end{itemize}

\subsection{Slab Regularization}

The slab component prevents coefficients from being too large:
\begin{equation}
\tilde{\lambda}_j = \frac{c \cdot \lambda_j}{\sqrt{c^2 + \tau^2 \lambda_j^2}}
\end{equation}

As $\tau \lambda_j \rightarrow \infty$:
\begin{equation}
\tilde{\lambda}_j \rightarrow c \cdot \frac{1}{\tau}
\end{equation}

This bounds the effective coefficient scale by $c$, preventing unrealistic effect sizes.

\newpage
\section{References}

\begin{enumerate}[leftmargin=1.5em, itemsep=0.5em]
    \item Piironen, J., \& Vehtari, A. (2017). Sparsity information and regularization in the horseshoe and other shrinkage priors. \textit{Electronic Journal of Statistics}, 11(2), 5018-5051.
    
    \item Carvalho, C. M., Polson, N. G., \& Scott, J. G. (2010). The horseshoe estimator for sparse signals. \textit{Biometrika}, 97(2), 465-480.
    
    \item George, E. I., \& McCulloch, R. E. (1993). Variable selection via Gibbs sampling. \textit{Journal of the American Statistical Association}, 88(423), 881-889.
    
    \item Park, T., \& Casella, G. (2008). The Bayesian Lasso. \textit{Journal of the American Statistical Association}, 103(482), 681-686.
    
    \item Hernán, M. A., \& Robins, J. M. (2020). \textit{Causal Inference: What If}. Boca Raton: Chapman \& Hall/CRC.
    
    \item Pearl, J. (2009). \textit{Causality: Models, Reasoning and Inference} (2nd ed.). Cambridge University Press.
    
    \item Westfall, J., \& Yarkoni, T. (2016). Statistically controlling for confounding constructs is harder than you think. \textit{PLoS ONE}, 11(3), e0152719.
\end{enumerate}

\newpage
\appendix
\section{Quick Reference Card}

\begin{table}[h]
\centering
\begin{tabular}{ll}
\toprule
\textbf{Method} & \textbf{When to Use} \\
\midrule
Regularized Horseshoe & Default choice; sparse effects expected \\
Finnish Horseshoe & Same as regularized; emphasizes slab \\
Spike-and-Slab & When hard selection probabilities needed \\
Bayesian LASSO & Many small effects; less sparse \\
\bottomrule
\end{tabular}
\end{table}

\begin{table}[h]
\centering
\begin{tabular}{lccc}
\toprule
\textbf{Variable Type} & \textbf{Selection OK?} & \textbf{Standard Prior?} & \textbf{Exclude?} \\
\midrule
Weather & \checkmark & \checkmark & -- \\
Economic indicators & \checkmark & \checkmark & -- \\
Minor calendar effects & \checkmark & \checkmark & -- \\
Distribution/ACV & -- & \checkmark & -- \\
Price & -- & \checkmark & -- \\
Competitor media & -- & \checkmark & -- \\
Mediators (awareness) & -- & -- & \checkmark \\
\bottomrule
\end{tabular}
\caption{Variable handling recommendations}
\end{table}

\section{Code Templates}

\subsection{Horseshoe with Excluded Confounders}

\begin{lstlisting}[language=Python]
# Split controls into selectable and non-selectable
confounders = ["distribution", "price", "competitor_media"]
precision_vars = [c for c in control_names if c not in confounders]

with pm.Model() as model:
    sigma = pm.HalfNormal("sigma", 0.5)
    
    # Standard priors for confounders (must include)
    beta_confounders = pm.Normal(
        "beta_confounders", 
        mu=0, sigma=0.5, 
        shape=len(confounders)
    )
    
    # Selection priors for precision variables
    config = VariableSelectionConfig(
        method=VariableSelectionMethod.REGULARIZED_HORSESHOE,
        horseshoe=HorseshoeConfig(expected_nonzero=3),
    )
    result = create_control_selection_prior(
        "beta_precision", len(precision_vars), n_obs, sigma, config
    )
    
    # Combine contributions
    control_contrib = (
        pm.math.dot(X_confounders, beta_confounders) +
        pm.math.dot(X_precision, result.beta)
    )
\end{lstlisting}

\end{document}